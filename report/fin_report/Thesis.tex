%% ----------------------------------------------------------------
%% Thesis.tex -- MAIN FILE (the one that you compile with LaTeX)
\documentclass[a4paper, 11pt, oneside]{Thesis}  % Use the "Thesis" style, based on the ECS Thesis style by Steve Gunn
\graphicspath{Figures/}  % Location of the graphics files (set up for graphics to be in PDF format)

% Include any extra LaTeX packages required
\usepackage[square, numbers, comma, sort&compress]{natbib}  % Use the "Natbib" style for the references in the Bibliography
\usepackage{verbatim}  % Needed for the "comment" environment to make LaTeX comments
\usepackage{vector}  % Allows "\bvec{}" and "\buvec{}" for "blackboard" style bold vectors in maths
\hypersetup{urlcolor=blue, colorlinks=true}  % Colours hyperlinks in blue, but this can be distracting if there are many links.

%% ----------------------------------------------------------------
\begin{document}
\frontmatter      % Begin Roman style (i, ii, iii, iv...) page numbering

% Set up the Title Page
\title  {Translating motion sequences to Vocab Sequences using 3D
Convolutional Networks}
\authors  {\texorpdfstring
            {\href{http://sabsathai.github.io/}{Pranav Sankhe}}
            {Pranav Sankhe}
            }
\addresses  {\groupname\\\deptname\\\univname}  % Do not change this here, instead these must be set in the "Thesis.cls" file, please look through it instead
\date       {\today}
\subject    {}
\keywords   {}

\maketitle
%% ----------------------------------------------------------------

\setstretch{1.3}  % It is better to have smaller font and larger line spacing than the other way round

% Define the page headers using the FancyHdr package and set up for one-sided printing

\fancyhead{}  % Clears all page headers and footers
\rhead{\thepage}  % Sets the right side header to show the page number
\lhead{}  % Clears the left side page header

\pagestyle{fancy}  % Finally, use the "fancy" page style to implement the FancyHdr headers

%% ----------------------------------------------------------------
% Declaration Page required for the Thesis, your institution may give you a different text to place here
%\Declaration

%{
%
%\addtocontents{toc}{\vspace{1em}}  % Add a gap in the Contents, for aesthetics
%
%We , AUTHOR NAME, declare that this thesis titled, `THESIS TITLE' and the work presented in it are my own. I confirm that:
%
%\begin{itemize} 
%\item[\tiny{$\blacksquare$}] This work was done wholly or mainly while in candidature for a research degree at this University.
% 
%\item[\tiny{$\blacksquare$}] Where any part of this thesis has previously been submitted for a degree or any other qualification at this University or any other institution, this has been clearly stated.
% 
%\item[\tiny{$\blacksquare$}] Where I have consulted the published work of others, this is always clearly attributed.
% 
%\item[\tiny{$\blacksquare$}] Where I have quoted from the work of others, the source is always given. With the exception of such quotations, this thesis is entirely my own work.
% 
%\item[\tiny{$\blacksquare$}] I have acknowledged all main sources of help.
% 
%\item[\tiny{$\blacksquare$}] Where the thesis is based on work done by myself jointly with others, I have made clear exactly what was done by others and what I have contributed myself.
%\\
%\end{itemize}
% 
% 
%Signed:\\
%\rule[1em]{25em}{0.5pt}  % This prints a line for the signature
% 
%Date:\\
%\rule[1em]{25em}{0.5pt}  % This prints a line to write the date
%}
%\clearpage  % Declaration ended, now start a new page

%% ----------------------------------------------------------------
% The "Funny Quote Page"
%\pagestyle{empty}  % No headers or footers for the following pages
%
%\null\vfill
%% Now comes the "Funny Quote", written in italics
%\textit{``Write a funny quote here.''}
%
%\begin{flushright}
%If the quote is taken from someone, their name goes here
%\end{flushright}
%
%\vfill\vfill\vfill\vfill\vfill\vfill\null
%\clearpage  % Funny Quote page ended, start a new page
%% ----------------------------------------------------------------

% The Abstract Page
\section {Abstract}  % Add the "Abstract" page entry to the Contents

%\addtocontents{toc}{\vspace{1em}}  % Add a gap in the Contents, for aesthetics
Around $5\%$ of the world population suffers from hearing loss. Engaging differently abled people who have hearing problems in everyday conversations requires action based language which we call as sign language. Deaf people use Sign Language to communicate with each other which is often native to the country. Japanese Sign Language is used by around 317000 people. However there are lack of interfaces to translate the Sign Language to \textbf{NL} to enable conversation with deaf people and the hearing ones. The hearing people can't grasp the sign langauge with the normal signing speed and it would greatly benefit the deaf people to have the translations of \textbf{NL} to sign language. Hence it is essential to have a bi-directional system which translates Sign Language to \textbf{NL} and vice-versa. 
However, \textbf{SL} is not a regular language, it doesn't use voice but actions of arms, fingers and facial expression which increases the complexity of the recognition and generation task. developed to recognize SL words or to generate SL animations generally do not account for all aspects of a signed sentence, such as facial expression, natural signing speed, transitions between words and temporal and spatial context information [1], what makes them incomplete and hard to interprete. Hence the recognition and generation of \textbf{SL} hasn't been implemented with good enough quality that it can be practically implemented. 
To represent the multi-dimensional aspects of SL and solve the issues related to its continuous movements, we investigate the use of deep Machine Learning (ML) models, useful for many domains. The translation of Sign Language to Vocab Sequences has been implemented here based on the neural machine translation architecture developed initially for natural language translation. 
The encoder of the sequence-to-sequence model is inspired from the models developed for action recognition(classification) tasks and modified accordingly for sequence prediction task. The analysis proves that with an extended dataset, the designed trainable model can be put to use to aid the deaf people. Over the course of this two month internship, we investigated two major models for translation of \textbf{SL} to \textbf{NL}. The overall framework of the two models is a sequence to sequence model designed for \textbf{NL} translation. The two models investigate the idea of encoding the motion data[SL] into a fixed representation which can be further used to predict the sentences in \textbf{NL}. \\ 

This report extensively elaborates on the data analysis, the model architectures, data processing for the input data and details of the implementation in \texttt{TensorFlow}. Special attention has been devoted to explain the code for further reproducibility.  



\clearpage  % Abstract ended, start a new page
%% ----------------------------------------------------------------


\setstretch{1.3}  % Reset the line-spacing to 1.3 for body text (if it has changed)

% The Acknowledgements page, for thanking everyone
%\acknowledgements{
%\addtocontents{toc}{\vspace{1em}}  % Add a gap in the Contents, for aesthetics
%
%The acknowledgements and the people to thank go here, don't forget to include your project advisor\ldots
%
%}
%\clearpage  % End of the Acknowledgements
%% ----------------------------------------------------------------

\pagestyle{fancy}  %The page style headers have been "empty" all this time, now use the "fancy" headers as defined before to bring them back


%% ----------------------------------------------------------------
%\lhead{\emph{Contents}}  % Set the left side page header to "Contents"
\tableofcontents  % Write out the Table of Contents

%% ----------------------------------------------------------------
%\lhead{\emph{List of Figures}}  % Set the left side page header to "List if Figures"
%\listoffigures  % Write out the List of Figures

%% ----------------------------------------------------------------
%\lhead{\emph{List of Tables}}  % Set the left side page header to "List of Tables"
%\listoftables  % Write out the List of Tables

%% ----------------------------------------------------------------
\setstretch{1.5}  % Set the line spacing to 1.5, this makes the following tables easier to read
\clearpage  % Start a new page
%\lhead{\emph{Abbreviations}}  % Set the left side page header to "Abbreviations"
\listofsymbols{ll}  % Include a list of Abbreviations (a table of two columns)
{
% \textbf{Acronym} & \textbf{W}hat (it) \textbf{S}tands \textbf{F}or \\
\textbf{SL}  & \textbf{S}ign \textbf{L}anguage  \\
\textbf{NL}  & \textbf{N}atural \textbf{L}anguage\\
}

%% ----------------------------------------------------------------
%\clearpage  % Start a new page
%\lhead{\emph{Physical Constants}}  % Set the left side page header to "Physical Constants"
%\listofconstants{lrcl}  % Include a list of Physical Constants (a four column table)
%{
%% Constant Name & Symbol & = & Constant Value (with units) \\
%Speed of Light & $c$ & $=$ & $2.997\ 924\ 58\times10^{8}\ \mbox{ms}^{-\mbox{s}}$ (exact)\\
%
%}

%% ----------------------------------------------------------------
%\clearpage  %Start a new page
%\lhead{\emph{Symbols}}  % Set the left side page header to "Symbols"
%\listofnomenclature{lll}  % Include a list of Symbols (a three column table)
%{
%% symbol & name & unit \\
%$a$ & distance & m \\
%$P$ & power & W (Js$^{-1}$) \\
%& & \\ % Gap to separate the Roman symbols from the Greek
%$\omega$ & angular frequency & rads$^{-1}$ \\
%}
%% ----------------------------------------------------------------
% End of the pre-able, contents and lists of things
% Begin the Dedication page

\setstretch{1.3}  % Return the line spacing back to 1.3

%\pagestyle{empty}  % Page style needs to be empty for this page
%\dedicatory{For/Dedicated to/To my\ldots}

%\addtocontents{toc}{\vspace{2em}}  % Add a gap in the Contents, for aesthetics


%% ----------------------------------------------------------------
\mainmatter	  % Begin normal, numeric (1,2,3...) page numbering
\pagestyle{fancy}  % Return the page headers back to the "fancy" style

% Include the chapters of the thesis, as separate files
% Just uncomment the lines as you write the chapters

\chapter{Data Analysis}

In order to get familiar with the dataset, we tried some pre-processing and visualizations of the data. The data analysis enabled us to select the data format for our architectures. 

\section{Variance based Analysis}
We had the motion data captured using the Vicon System which was converted to the \texttt{BVH} format. The BVH format ahd the hierarchical structure which represented the human skeletal structure. The joints were constructed and ordered in the form of kinematic chains and each joint had six motion parameters. The first 3 represented the lengths of the individual segments and the other three denote the rotation angles of the joints. We performed varaince based analysis to determine the motion of each joint and which joint is contributing substantially to the overall motion.  
Fig:1.1 plots variances of all markers througout the signing of one sentence. The indvidual variances of each of the 6 motion parameters are plotted seperately.  
\begin{figure}[h!]
	\includegraphics[width=\linewidth]{./images/variance_1.png}
	\caption{ Variance of individual joints }
	\label{fig:1.1}
\end{figure}

\section{C3D data}

The motion capture data was captured using the Vicon system which gives output in the \texttt{C3D} format. The \texttt{C3D} format basically is point cloud data. We created a visualization of the C3D data which shows how  

 % Introduction
\newpage

\chapter{Two Stream RNN CNN based Encoder}

\section{Architecture}

\section{Input Data}

\section{Pre-Training}

\section{Results}
 % Background Theory 

\chapter{Optical Flow Computation}

\section{What is Optical Flow}

\section{Motivation to use Optical Flow}

\section{Optical Flow Code}


 % Experimental Setup

\chapter{Final Model}

\section{Initial Architecture}

\section{ResNet based Architecture}

\section{Training}

\section{Inference}


 % Experiment 1

\chapter{Future Work}

 % Experiment 2

\input{Chapters/Chapter6} % Results and Discussion

%\input{Chapters/Chapter7} % Conclusion

%% ----------------------------------------------------------------
% Now begin the Appendices, including them as separate files

\addtocontents{toc}{\vspace{2em}} % Add a gap in the Contents, for aesthetics

\appendix % Cue to tell LaTeX that the following 'chapters' are Appendices

\chapter{Pre-Training}

\chapter{Sequence to Sequence Model}






	% Appendix Title

%\input{Appendices/AppendixB} % Appendix Title

%\input{Appendices/AppendixC} % Appendix Title

\addtocontents{toc}{\vspace{2em}}  % Add a gap in the Contents, for aesthetics
\backmatter

%% ----------------------------------------------------------------
\label{Bibliography}
\lhead{\emph{Bibliography}}  % Change the left side page header to "Bibliography"
\bibliographystyle{unsrtnat}  % Use the "unsrtnat" BibTeX style for formatting the Bibliography
\bibliography{Bibliography}  % The references (bibliography) information are stored in the file named "Bibliography.bib"

\end{document}  % The End
%% ----------------------------------------------------------------