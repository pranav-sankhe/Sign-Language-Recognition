\chapter{Data Analysis}

In order to get familiar with the dataset, we tried some pre-processing and visualizations of the data. The data analysis enabled us to select the data format for our architectures. 

\section{Variance based Analysis}
We had the motion data captured using the Vicon System which was converted to the \texttt{BVH} format. The BVH format ahd the hierarchical structure which represented the human skeletal structure. The joints were constructed and ordered in the form of kinematic chains and each joint had six motion parameters. The first 3 represented the lengths of the individual segments and the other three denote the rotation angles of the joints. We performed varaince based analysis to determine the motion of each joint and which joint is contributing substantially to the overall motion.  
Fig:1.1 plots variances of all markers througout the signing of one sentence. The indvidual variances of each of the 6 motion parameters are plotted seperately.  
\begin{figure}[h!]
	\includegraphics[width=\linewidth]{./images/variance_1.png}
	\caption{ Variance of individual joints }
	\label{fig:1.1}
\end{figure}

\section{C3D data}

The motion capture data was captured using the Vicon system which gives output in the \texttt{C3D} format. The \texttt{C3D} format basically is point cloud data. We created a visualization of the C3D data which shows how  

